% Options for packages loaded elsewhere
\PassOptionsToPackage{unicode}{hyperref}
\PassOptionsToPackage{hyphens}{url}
%
\documentclass[
]{article}
\usepackage{amsmath,amssymb}
\usepackage{iftex}
\ifPDFTeX
  \usepackage[T1]{fontenc}
  \usepackage[utf8]{inputenc}
  \usepackage{textcomp} % provide euro and other symbols
\else % if luatex or xetex
  \usepackage{unicode-math} % this also loads fontspec
  \defaultfontfeatures{Scale=MatchLowercase}
  \defaultfontfeatures[\rmfamily]{Ligatures=TeX,Scale=1}
\fi
\usepackage{lmodern}
\ifPDFTeX\else
  % xetex/luatex font selection
\fi
% Use upquote if available, for straight quotes in verbatim environments
\IfFileExists{upquote.sty}{\usepackage{upquote}}{}
\IfFileExists{microtype.sty}{% use microtype if available
  \usepackage[]{microtype}
  \UseMicrotypeSet[protrusion]{basicmath} % disable protrusion for tt fonts
}{}
\makeatletter
\@ifundefined{KOMAClassName}{% if non-KOMA class
  \IfFileExists{parskip.sty}{%
    \usepackage{parskip}
  }{% else
    \setlength{\parindent}{0pt}
    \setlength{\parskip}{6pt plus 2pt minus 1pt}}
}{% if KOMA class
  \KOMAoptions{parskip=half}}
\makeatother
\usepackage{xcolor}
\usepackage[margin=1in]{geometry}
\usepackage{graphicx}
\makeatletter
\def\maxwidth{\ifdim\Gin@nat@width>\linewidth\linewidth\else\Gin@nat@width\fi}
\def\maxheight{\ifdim\Gin@nat@height>\textheight\textheight\else\Gin@nat@height\fi}
\makeatother
% Scale images if necessary, so that they will not overflow the page
% margins by default, and it is still possible to overwrite the defaults
% using explicit options in \includegraphics[width, height, ...]{}
\setkeys{Gin}{width=\maxwidth,height=\maxheight,keepaspectratio}
% Set default figure placement to htbp
\makeatletter
\def\fps@figure{htbp}
\makeatother
\setlength{\emergencystretch}{3em} % prevent overfull lines
\providecommand{\tightlist}{%
  \setlength{\itemsep}{0pt}\setlength{\parskip}{0pt}}
\setcounter{secnumdepth}{-\maxdimen} % remove section numbering
% definitions for citeproc citations
\NewDocumentCommand\citeproctext{}{}
\NewDocumentCommand\citeproc{mm}{%
  \begingroup\def\citeproctext{#2}\cite{#1}\endgroup}
\makeatletter
 % allow citations to break across lines
 \let\@cite@ofmt\@firstofone
 % avoid brackets around text for \cite:
 \def\@biblabel#1{}
 \def\@cite#1#2{{#1\if@tempswa , #2\fi}}
\makeatother
\newlength{\cslhangindent}
\setlength{\cslhangindent}{1.5em}
\newlength{\csllabelwidth}
\setlength{\csllabelwidth}{3em}
\newenvironment{CSLReferences}[2] % #1 hanging-indent, #2 entry-spacing
 {\begin{list}{}{%
  \setlength{\itemindent}{0pt}
  \setlength{\leftmargin}{0pt}
  \setlength{\parsep}{0pt}
  % turn on hanging indent if param 1 is 1
  \ifodd #1
   \setlength{\leftmargin}{\cslhangindent}
   \setlength{\itemindent}{-1\cslhangindent}
  \fi
  % set entry spacing
  \setlength{\itemsep}{#2\baselineskip}}}
 {\end{list}}
\usepackage{calc}
\newcommand{\CSLBlock}[1]{\hfill\break\parbox[t]{\linewidth}{\strut\ignorespaces#1\strut}}
\newcommand{\CSLLeftMargin}[1]{\parbox[t]{\csllabelwidth}{\strut#1\strut}}
\newcommand{\CSLRightInline}[1]{\parbox[t]{\linewidth - \csllabelwidth}{\strut#1\strut}}
\newcommand{\CSLIndent}[1]{\hspace{\cslhangindent}#1}
\usepackage{fontspec}
\setmainfont{Fira Sans Condensed}
\ifLuaTeX
  \usepackage{selnolig}  % disable illegal ligatures
\fi
\usepackage{bookmark}
\IfFileExists{xurl.sty}{\usepackage{xurl}}{} % add URL line breaks if available
\urlstyle{same}
\hypersetup{
  hidelinks,
  pdfcreator={LaTeX via pandoc}}

\author{}
\date{\vspace{-2.5em}}

\begin{document}

Revista Desarrollos en ciencias sociales computacionales Vol. 0 ( 0 ),
0-0

mincaqdasr. Herramienta minimalista para anotación y codificación de
documentos

mincaqdasr. A Minimalist Tool for Document Annotation and Coding

Gastón Becerra ( Universidad de Flores, Argentina )
\href{mailto:gaston.becerra@uflouniversidad.edu.ar}{\nolinkurl{gaston.becerra@uflouniversidad.edu.ar}}
https://orcid.org/0009-0006-9160-9744

Juan Pablo López-Alurralde ( Universidad de Flores, Argentina )
\href{mailto:juan.pablo.lopez@uflouniversidad.edu.ar}{\nolinkurl{juan.pablo.lopez@uflouniversidad.edu.ar}}
https://orcid.org/0009-0006-9160-9744

Resumen: mincaqdasr es una herramienta minimalista para el análisis
cualitativo (CAQDAS), diseñada puntualmente para codificar o anotar
corpus de documentos mediante una interfaz web. mincaqdasr genera un
archivo .json que incluye todos los elementos del proyecto, (documentos,
fragmentos, anotaciones, códigos, etc.) eliminando la necesidad de
gestionar carpetas adicionales. mincaqdasr puede ser invocado desde R o
de manera autónoma con cualquier navegador que tenga JavaScript
habilitado.

Palabras clave: CAQDAS, R, JavaScript, software para análisis
cualitativo de datos asistido por computadora, investigación cualitativa

Abstract: mincaqdasr is a minimalist tool for qualitative analysis
(CAQDAS), specifically designed for coding or annotating document
corpora through a web interface. mincaqdasr generates a .json file that
includes all project elements (documents, fragments, annotations, codes,
etc.), eliminating the need to manage additional folders. mincaqdasr can
be invoked from R or used independently with any browser that supports
JavaScript.

Keywords: CAQDAS, R, JavaScript, computer-assisted qualitative data
analysis software, qualitative research

Recibido: 2024-11-14 \textbar{} Aceptado: 0000-00-00

Recurso/Resources

URL

Repositorio/Repository

https://github.com/gastonbecerra/mincaqdasr

Licencia/Licence

https://github.com/gastonbecerra/mincaqdasr?tab=GPL-3.0-1-ov-file\#readme

Lenguajes/Languages Formatos/Formats

R

JavaScript

\subsection{Justificación}\label{justificaciuxf3n}

mincaqdasr es una herramienta para análisis cualitativo (CAQDAS),
diseñada específicamente para asistir en la tarea de codificar o anotar
un corpus de documentos breves a través de una interfaz web. En
investigación cualitativa, este proceso de anotación es conocido como
``codificación'', definido como ``\ldots{} una palabra o frase breve que
simbólicamente asigna un atributo sintético, saliente, esencial y/o
evocativo a una porción de datos basados en el lenguaje'' (Saldaña,
2016, p. 3). Este proceso suele ser uno de los primeros pasos en un
diseño espiralado o por etapas, donde los códigos se integran en temas y
argumentos para conceptualizar y desarrollar una narrativa teórica
(Auerbach \& Silverstein, 2003).

La herramienta ofrece una interfaz gráfica sencilla que permite
identificar y resaltar fragmentos de texto, crear e imputar códigos, y
registrar memos o comentarios. El resultado de la codificación se
exporta en formato JSON (JavaScript Object Notation), un estándar que
puede ser manipulado desde cualquier lenguaje de análisis de datos, como
R o Python. De este modo, mincaqdasr promueve la transparencia y la
colaboración en la investigación cualitativa.

Construida como una aplicación web autónoma, mincaqdasr puede ejecutarse
en cualquier navegador que tenga JavaScript habilitado. Su diseño
minimalista se centra exclusivamente en las tareas esenciales para la
codificación, dejando al investigador la flexibilidad de gestionar otras
etapas del análisis cualitativo, como la limpieza de datos, el manejo de
documentos o la construcción de tablas y redes temáticas, utilizando los
flujos de trabajo que considere más convenientes.

El corpus se puede cargar como un vector simple de texto, eliminando la
necesidad de crear carpetas o paquetes (bundles) como en otras
herramientas. Para facilitar el trabajo en R, mincaqdasr incluye
funciones para iniciar la interfaz como una shinyapp y para manipular el
archivo JSON resultante de la codificación.

\subsection{Estructura de los datos}\label{estructura-de-los-datos}

Los datos se guardan en un archivo JSON que incluye 4 elementos:

\begin{itemize}
\tightlist
\item
  documentos (como un vector de texto);
\item
  códigos (un vector de texto);
\item
  anotaciones a nivel documento, referenciando uno o varios códigos y un
  memo del anotador;
\item
  anotaciones a nivel fragmento, referenciando uno o varios códigos y un
  memo del anotador;
\end{itemize}

Incluimos dos ejemplos breves de codificaciones en formato JSON. El
primero retoma fragmentos del trabajo de M. Zizi en el proyecto ``The
Yeshiva University Fatherhood Project'', que Auerbach y Silverstein
(2003) presentan en su libro, ilustrando paso a paso las tareas clave
del análisis cualitativo: desde la lectura y selección de fragmentos
hasta la asignación de códigos y su elaboración conceptual. Con
mincaqdasr, es posible registrar todas estas tareas, excepto la última.

El segundo ejemplo contiene fragmentos de noticias de la prensa
argentina sobre ``big data''. En este caso, no se asignan códigos a
fragmentos específicos; en su lugar, cada documento se clasifica con una
etiqueta ``positivo'' o ``negativo'', según mencione una aplicación
beneficiosa o un riesgo. Otras tareas de análisis con este corpus se
detallan en otro material propio (Becerra \& López-Alurralde, 2024).

\subsection{Otras herramientas}\label{otras-herramientas}

Existen varias herramientas comerciales para asistir en el análisis
cualitativo (CAQDAS). Las más conocidas son NVivo, Maxda o Atlas.ti, que
incluyen varias funciones de manejo de documentos y análisis, y que
trabajan con varios formatos de documentos, no sólo textuales. A
diferencia de estas, mincaqdasr es gratuita y de código abierto, y se
limita a texto plano y no cubre funciones más allá de las necesarias
para anotar o codificar, lo que puede ser conveniente cuando el
investigador maneja su flujo de trabajo en un lenguaje de análisis de
datos como R.

En R se encuentra disponible el package qcoder (Duckles et al., 2024)
que también permite la codificación a través de una interfaz gráfica
que, además, permite la exploración de códigos y fragmentos. Sin
embargo, tiene un manejo del proyecto y sus archivos mucho más
estructurado, de modo que puede resultar más dificil de integrar en un
código más amplio. No conocemos otros proyectos activos --y se debe
advertir que qcoder no parece haber tenido actualizaciones en los
últimos años-- que cumplan funciones de CAQDAS.

\subsection{Referencias}\label{referencias}

\phantomsection\label{refs}
\begin{CSLReferences}{1}{0}
\bibitem[\citeproctext]{ref-auerbach2003}
Auerbach, C. F., \& Silverstein, L. B. (2003). \emph{Qualitative data:
An introduction to coding and analysis}. NYU Press.

\bibitem[\citeproctext]{ref-cursor}
Becerra, G., \& López-Alurralde, J. P. (2024). \emph{Curso introductorio
a r para las ciencias sociales}.
\url{https://bookdown.org/gaston_becerra/curso-intro-r/}

\bibitem[\citeproctext]{ref-qcoder2024}
Duckles, B., Sholler, D., Draper, J., \& Laderas, T. (2024).
\emph{Qcoder: Lightweight qualitative coding}.
\url{https://github.com/ropenscilabs/qcoder}

\bibitem[\citeproctext]{ref-saldauxf1a2016}
Saldaña, J. (2016). \emph{The coding manual for qualitative
researchers}. SAGE Publications.

\end{CSLReferences}

Gastón Becerra, Juan Pablo López-Alurralde ( 2024 ). mincaqdasr.
Herramienta minimalista para anotación y codificación de documentos .
Desarrollos en ciencias sociales computacionales 0 ( 0 ) , 0-0

Revista Desarrollos en Ciencias Sociales Computacionales \textbar{}
ISSN: en trámite https://revistadesarrollos.uflo.edu.ar/ Licenciatura en
Sociología / Facultad de Psicología y Ciencias Sociales / Universidad de
Flores, Argentina

\end{document}
